\documentclass[a4paper, 11pt, titlepage]{article}
\usepackage{fancyhdr}
\usepackage{graphicx}
\usepackage{imakeidx}
\usepackage{makeidx}
\usepackage{mathtools}
\usepackage[spanish]{babel}
\usepackage{eurosym}
\usepackage{hyperref}
\usepackage{amssymb}
\usepackage{listings}
\usepackage{xcolor}

% \setcounter{secnumdepth}{5}
% \setcounter{tocdepth}{5}

\title{Sistemas expertos en combinación con lógica difusa}
\author{Francisco Javier Balón Aguilar}

\begin{document}

\maketitle
\renewcommand{\contentsname}{Índice}
\tableofcontents
\newpage

Un sistema distribuido es aquel sistema coordinado cuyos componentes se comunican a través de una red mediante mensajes, siendo el proceso totalmente transparente para el usuario. Internet puede ser considerado como el ejemplo de sistema distribuido más amplio, pero también puede ser cualquier otra red privada, P2P, etc. Para coordinar estos sistemas, son necesarios lo que se conocen como sistemas operativos distribuidos que garanticen el uso adecuado de los recursos, a nivel de hardware, software y de acceso a datos. Además, controlan la serialización de los procesos, la coherencia en la actualización de los datos (sistemas transaccionales), la actualización de los relojes de los componentes mediante algoritmos de sincronización, la seguridad, y en general el correcto funcionamiento del sistema.

Para facilitar el funcionamiento dentro de un sistema distribuido, se deben tener en cuenta las siguientes características:


% BIBLIOGRAFÍA Y REFERENCIAS
\newpage
\begin{thebibliography}{X}
    \bibitem{} Desarrollo de un sistema expertocon lógica difusa, Jorge Franco Herrera y Angélica Franco Arias \\ \url{https://ingsistycomp.files.wordpress.com/2017/09/proyecto-1-sistema-experto-difuso.pdf}
    \bibitem{} Diseño de un Sistema Experto Difuso para la Determinación de la Densidad de Corriente en una Planta de Cromado, Carolina V. Ponce y Bayron Rojas \\ \url{https://scielo.conicyt.cl/scielo.php?script=sci_arttext&pid=S0718-07642019000200157}
    \bibitem{} Modelo basado en Lógica Difusa para el Diagnóstico Cognitivo del Estudiante \\ \url{https://scielo.conicyt.cl/scielo.php?script=sci_arttext&pid=S0718-50062012000100003}
    \bibitem{} Sistemas Expertos y Lógica Difusa \\ \url{http://catarina.udlap.mx/u_dl_a/tales/documentos/lmt/maza_c_ac/capitulo2.pdf}
    \bibitem{} Sistema de Control Difuso para Unidades de Cuidados Intensivos (UCI), Jefferson Steven Soto Medellín \\ \url{https://repository.ucatolica.edu.co/bitstream/10983/1278/1/Sistemas%20de%20Control%20Difuso%20para%20Unidades%20de%20Cuidado%20Intensivo%20(Trabajo%20Final)%20701429%20Nuevo.pdf}
\end{thebibliography}

\end{document}
