\documentclass[a4paper, 11pt, titlepage]{article}
\usepackage{fancyhdr}
\usepackage{graphicx}
\usepackage{imakeidx}
\usepackage{makeidx}
\usepackage{mathtools}
\usepackage[spanish]{babel}
\usepackage{eurosym}
\usepackage{hyperref}
\usepackage{amssymb}
\usepackage{listings}
\usepackage{xcolor}
\usepackage{amsmath}

\title{Le matemática de Pokémon}
\author{Francisco Javier Balón Aguilar}

\begin{document}

\maketitle
\renewcommand{\contentsname}{Índice}
\tableofcontents
\newpage

\section{Introducción}\label{introduccion}

\section{Cálculo de estadísticas}

    Las características base de un Pokémon son valores de referencia que fundamentan 
    las estadísticas de éstos, las cuales varían en cada especie y son inalterables; sirviendo 
    como base para el cálculo de sus estadísticas individuales.

    El listado de características es:

    \begin{itemize}
        \item \textbf{Puntos de salud (PS)}. Es la condición física del Pokémon, representada 
        con un valor numérico. El Pokémon se considera debilitado cuando sus PS llegan a 0, que 
        se buscará mediante los ataques del oponente en combate, los cambios de estado (como 
        el veneno o la quemadura) o climatología adversa (como el granizo o la tormenta de arena).
        
        \item \textbf{Ataque}. Representa la fuerza natural del Pokémon al realizar un ataque físico, 
        considerándose ataque físico a todo movimiento donde el Pokémon cause daño al hacer contacto 
        con el adversario, o en el que se requiera usar la fuerza física.

        Hasta la III generación, la distinción entre ataque físico y especial se hacía por medio 
        del tipo del ataque, siendo ataques físicos los tipos \textit{fantasma}, \textit{lucha}, 
        \textit{normal}, \textit{tierra}, \textit{roca}, \textit{bicho}, \textit{veneno}, 
        \textit{acero} y \textit{volador}. A partir de la IV generación depende de cada movimiento 
        particular.

        Sobre las alteraciones de estado, las quemaduras están estrechamente relacionadas con el 
        ataque físico, pues si un Pokémon se encuentra quemado su ataque se disminuirá en un $50\%$.
        
        \item \textbf{Defensa}. Representa la resistencia natural de un Pokémon ante los movimientos 
        físicos.
        
        \item \textbf{Ataque especial}. Representa la fuerza con la que un Pokémon realiza un movimiento 
        especial, considerándose éstos a los movimientos en los que se realiza un ataque a distancia, o 
        un ataque en el que se libera la energía del atacante.

        Hasta la III generación, los ataques especiales eran de los tipos \textit{agua}, \textit{hielo},
        \textit{fuego}, \textit{planta}, \textit{eléctrico}, \textit{dragón}, \textit{psíquico} y 
        \textit{siniestro}.
        
        \item \textbf{Defensa especial}. Representa la resistencia a los movimientos especiales.
        
        \item \textbf{Especial}. En la I generación no existían las características \textbf{ataque especial}
        y \textbf{defensa especial}. En su lugar se encontraba la característica \textbf{especial} que hacía 
        la función de ambas.
        
        \item \textbf{Velocidad}. Es la propiedad del Pokémon de atacar, antes o después, que el oponente.
        
        Sobre las alteraciones de estado, la parálisis está estrechamente relacionada con la velocidad, pues 
        si un Pokémon se encuentra paralizado, su velocidad se reducirá en un $75\%$.
        
        \item \textbf{Precisión}. Es una estadística temporal, únicamente modificable en combate. Su 
        utilidad se basa en la probabilidad de que el Pokémon acierte o falle al usar un movimiento, contando 
        éstos con una precisión determinada (para ver los cálculos con más detalle, ver sección \ref{precision}).

        \item \textbf{Evasión}. Es una estadística temporal, únicamente modificable en combate. Su
        utilidad es inversa a la \textbf{precisión}, siendo la probabilidad de que el Pokémon reciba o evite 
        el ataque de un movimiento dirigido a él mismo. La evasión se aplica únicamente a los Pokémon, no a los 
        movimientos (para ver los cálculos con más detalle, ver sección \ref{precision}).
    \end{itemize}

    En las generaciones I y II, hay dos ecuaciones que determinan las estadísticas 
    de un Pokémon. El cálculo de PS es distinto al cálculo del resto de estadísticas.

    \[
        ps = [\frac{((2 \times b) + iv + [\frac{ev}{4}]) \times n}{100}] + n + 10
    \]

    \[
        c = [\frac{(b + iv) \times 2 + [\frac{\sqrt{ev}}{4}] \times n}{100}] + 5    
    \]

    Donde $c$ es cualquier característica, a excepción de puntos de salud ($ps$).
    los valores individuales relacionados con la característica y $ev$ a los puntos 
    de esfuerzo.

    En las generaciones posteriores, las fórmulas se calculan de la siguiente forma:

    \[
        ps = [\frac{((2 \times b) + iv + [\frac{ev}{4}]) \times n}{100}] + n + 10    
    \]

    \[
        c = [([\frac{((2 \times b) + iv + [\frac{ev}{4}]) \times n}{100}] + 5) \times nat]    
    \]

    Donde la nueva variable $nat$ hace referencia a la naturaleza, que es un modificador 
    añadido en la III generación que, depende de la naturaleza que posee el sujeto afecta a 
    la característica de forma desfavorable ($0.9$, neutra ($1.0$) o favorable ($1.1$)).

\section{Cálculo de poder oculto}

\section{Las matemáticas del combate}

    \subsection{Cálculo de daño}

    \subsection{Cálculo de huida}

    \subsection{Cálculo de precisión}\label{precision}

        Que un movimiento cumpla su objetivo y sea certero depende de la \textbf{precisión} y 
        la \textbf{evasión}, que se determina mediante la fórmula:

        \[
            a = pb \times \frac{p}{e}    
        \]

        Donde $pb$ corresponde con la precisión base del movimiento utilizado, en tanto por 
        uno (dividido el valor bruto entre $100$). $p$ referencia a la precisión del Pokémon 
        usuario del movimiento y $e$ referencia a la evasión del Pokémon rival, receptor del 
        movimiento. 
        
        Tanto $p$ como $e$ se calculan mediante el número de iteraciones, cuyo límites se 
        encuentran en el rango $[-6,6]$, que aumentan o disminuyen la precisión y/o la evasión; 
        partiendo siempre de $0$, valor que puede ser modificado en combate, siendo el cálculo  
        de $p$ (precisión) y de $e$ (evasión) inversamente proporcionales entre sí:

        \[
            p = pb
            \begin{pmatrix}
                9/3 \\ 
                8/3 \\ 
                7/3 \\ 
                6/3 \\ 
                5/3 \\ 
                4/3 \\ 
                3/3 \\ 
                3/4 \\ 
                3/5 \\ 
                3/6 \\ 
                3/7 \\ 
                3/8 \\ 
                3/9 \\ 
            \end{pmatrix}
            = 
            \begin{pmatrix}
                3pb \\ 
                2,\widehat{6}pb \\ 
                2,\widehat{3}pb \\ 
                2pb \\ 
                1,\widehat{6}pb \\ 
                1,\widehat{3}pb \\ 
                pb \\ 
                0,75pb \\ 
                0,6pb \\ 
                0,5pb \\ 
                0,4285pb \\ 
                0,375pb \\ 
                0,\widehat{3}pb \\ 
            \end{pmatrix}
            e = pb
            \begin{pmatrix}
                3/9 \\ 
                3/8 \\ 
                3/7 \\ 
                3/6 \\ 
                3/5 \\ 
                3/4 \\ 
                3/3 \\ 
                4/3 \\ 
                5/3 \\ 
                6/3 \\ 
                7/3 \\ 
                8/3 \\ 
                9/3 \\ 
            \end{pmatrix}
            = 
            \begin{pmatrix}
                0,\widehat{3}pb \\ 
                0,375pb \\ 
                0,4285pb \\ 
                0,5pb \\ 
                0,6pb \\ 
                0,75pb \\ 
                pb \\ 
                1,\widehat{3}pb \\ 
                1,\widehat{6}pb \\ 
                2pb \\ 
                2,\widehat{3}pb \\ 
                2,\widehat{6}pb \\ 
                3pb \\ 
            \end{pmatrix}
            \begin{bmatrix}
                i_{+6} \\
                i_{+5} \\
                i_{+4} \\
                i_{+3} \\
                i_{+2} \\
                i_{+1} \\
                i_{0} \\
                i_{-1} \\
                i_{-2} \\
                i_{-3} \\
                i_{-4} \\
                i_{-5} \\
                i_{-6} \\
            \end{bmatrix}
        \]

        Si tras el cálculo general $a$ es mayor que $1$, el movimiento acertará.

\end{document}