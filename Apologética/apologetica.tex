\documentclass[11pt, oneside, titlepage]{article}
\usepackage{fancyhdr}
\usepackage{graphicx}
\usepackage{imakeidx}
\usepackage{makeidx}
\usepackage{mathtools}
\usepackage[spanish]{babel}
\usepackage{graphicx}
\usepackage{dsfont}
 
\title{\textbf{Apologética Científica}}
\author{Francisco Javier Balón Aguilar}
%\date{}

\usepackage{amsmath}
\begin{document}
\maketitle
% Índice de contenidos
\tableofcontents
\newpage


    El origen de la vida ha sido un motivo permanente de reflexión por parte del pensamiento especulativo 
    de todos los tiempos.
    
    Como es bien sabido, desde la más remota antigüedad y hasta hace relativamente poco, apenas un siglo, 
    se pensaba que la vida podía originarse en forma espontánea, a partir de la materia inanimada. Toda 
    experiencia parecía confirmarlo, siendo una conclusión perfectamente razonable y lógica de acuerdo a 
    los métodos de observación disponibles y a los conocimientos de la época. También es una conclusión 
    <<perfectamente equivocada>>, como el Dr. Raúl Osvaldo Leguizamón apunta, como hoy sabemos tras Louis 
    Pasteur, quien demostró --definitivamente-- que, bajo las condiciones actuales de la naturaleza, no 
    existe generación de vida en forma espontánea a partir de materia inanimada.

    ¿De dónde provino, pues, la primera manifestación de vida? Este es uno de los problemas más apasionantes 
    de la Biología en este momento histórico, incluso siendo una de las aspiraciones científicas detrás del 
    proyecto espacial.

    Es conveniente aclarar que, a pesar de los experimentos de Pasteur, actualmente muchos científicos
    siguen creyendo en la generación espontánea de la vida a partir de la materia inanimada, pero con una 
    nueva denominación más <<científica y elegante>>; \textit{arquebiopoyesis} o \textit{biogénesis primitiva}, 
    afirmándose que los experimentos de Pasteur sólo demostraron que la generación espontánea no ocurre ahora,
    pero no, que no haya ocurrido en el pasado bajo otras circunstancias.



% En torno al origen de la vida - Raúl Osvaldo Leguizamón

\end{document}
