\documentclass[a4paper, 11pt, titlepage]{article}
\usepackage{fancyhdr}
\usepackage{graphicx}
\usepackage{imakeidx}
\usepackage{makeidx}
\usepackage{mathtools}
\usepackage[spanish]{babel}
\usepackage{eurosym}
\usepackage{hyperref}
\usepackage{amssymb}
\usepackage{listings}
\usepackage{xcolor}

\title{Le técnica de Pokémon}
\author{Francisco Javier Balón Aguilar}

\begin{document}

\maketitle
\renewcommand{\contentsname}{Índice}
\tableofcontents
\newpage

\section{Introducción}\label{introduccion}

\section{Cálculo de estadísticas}

    En las generaciones I y II, hay dos ecuaciones que determinan las estadísticas 
    de un Pokémon. El cálculo de PS es distinto al cálculo del resto de estadísticas.

    \[
        ps = [\frac{((2 \times b) + iv + [\frac{ev}{4}]) \times n}{100}] + n + 10
    \]

    \[
        c = [\frac{(b + iv) \times 2 + [\frac{\sqrt{ev}}{4}] \times n}{100}] + 5    
    \]

    Donde $c$ es cualquier característica, a excepción de puntos de salud ($ps$).
    los valores individuales relacionados con la característica y $ev$ a los puntos 
    de esfuerzo.

    En las generaciones posteriores, las fórmulas se calculan de la siguiente forma:

    \[
        ps = [\frac{((2 \times b) + iv + [\frac{ev}{4}]) \times n}{100}] + n + 10    
    \]

    \[
        c = [([\frac{((2 \times b) + iv + [\frac{ev}{4}]) \times n}{100}] + 5) \times nat]    
    \]

    Donde la nueva variable $nat$ hace referencia a la naturaleza, que es un modificador 
    añadido en la III generación que, depende de la naturaleza que posee el sujeto afecta a 
    la característica de forma desfavorable ($0.9$, neutra ($1.0$) o favorable ($1.1$)).

\section{Cálculo de poder oculto}

\section{Combate}

    \subsection{Cálculo de daño}

    \subsection{Cálculo de huida}

    \subsection{Cálculo de precisión}

\end{document}