\documentclass[a4paper, 11pt, titlepage]{article}
\usepackage{fancyhdr}
\usepackage{graphicx}
\usepackage{imakeidx}
\usepackage{makeidx}
\usepackage{mathtools}
\usepackage[spanish]{babel}
\usepackage{eurosym}
\usepackage{hyperref}
\usepackage{amssymb}
\usepackage{listings}
\usepackage{xcolor}
\usepackage{amsmath}

\title{SAV Hacking de Pokémon Rojo Fuego}
\author{Francisco Javier Balón Aguilar}

\begin{document}

\maketitle
\renewcommand{\contentsname}{Índice}
\tableofcontents
\newpage

\section{Programación en C para la III generación, desde la perspectiva de la ROM <<FireRed>>}

    La primera distinción a tener en cuenta, es que la programación en C para GBA no es lo 
    mismo que la programación en C En general. Cuando programamos específicamente para el 
    hardware de GBA, estamos trabajando en un entorno de nivel extremadamente bajo, basado en 
    registros de E/S y escribiendo datos en búferes de datos VRAM, etc. Lo que puede estar 
    acostumbrado a escribir son programas que llaman a varias funciones de bibliotecas como 
    <stdio.h>. Este tipo de conceptos de "nivel superior" realmente no hacen una aparición. 
    Las cosas más comparables a algo como stdio.h sería las funciones de BIOS del GBA, de 
    las que hablaré más adelante.

    Finalmente, la segunda y última distinción a hacer es que la programación para el GBA 
    en general no es lo mismo que la programación para un hack ROM. Cuando se trabaja con 
    una ROM precompilada, nos vemos obligados a trabajar dentro de los confines de la base 
    de código existente, y tenemos que trabajar alrededor de los sistemas en su lugar por 
    el motor predeterminado. Si ya estás familiarizado con cómo codificar en C, entonces es 
    posible que puedas leer el código, pero es posible que te cueste entender lo que 
    realmente hace. Para remediar esta situación, sugiero una sólida comprensión de la 
    piratería ASM en la generación III ROM hacks. Específicamente Gen III ROM hacks, porque 
    usted debe tener una buena comprensión de cómo FireRed y otros trabajan inorder para 
    escribir buen código para explotar lo que ya está allí.


\end{document}