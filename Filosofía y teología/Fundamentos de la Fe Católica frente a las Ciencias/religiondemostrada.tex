\documentclass[11pt, twosides, titlepage]{article}
\usepackage{fancyhdr}
\usepackage{graphicx}
\usepackage{imakeidx}
\usepackage{makeidx}
\usepackage{mathtools}
\usepackage[spanish]{babel}
\usepackage{graphicx}
\usepackage{amssymb}
\usepackage{listings}
\usepackage{xcolor}
\usepackage{hyperref}

% CODE C
\usepackage{amssymb}
\usepackage{listings}
\usepackage{xcolor}
\definecolor{textblue}{rgb}{.2,.2,.7}
\definecolor{textred}{rgb}{0.54,0,0}
\definecolor{textgreen}{rgb}{0,0.43,0}
\lstset{
numberstyle=\tiny, 
stepnumber=1,
numbersep=5pt, 
tabsize=4,
basicstyle=\ttfamily,
keywordstyle=\color{textblue},
commentstyle=\color{textred},   
stringstyle=\color{textgreen},
frame=none,                    
columns=fullflexible,
keepspaces=true,
xleftmargin=\parindent,
showstringspaces=false}

\lstset{basicstyle=\ttfamily,
    showstringspaces=false,
    commentstyle=\color{red},
    keywordstyle=\color{blue}
}

\setcounter{secnumdepth}{5}
\setcounter{tocdepth}{5}

\title{{\scshape\Huge Fundamentos de la Fe Católica a la luz de la Ciencia \par}}
\author{Francisco Javier Balón Aguilar}

\begin{document}
\maketitle
\renewcommand{\contentsname}{Índice de contenidos} % Nombre dado al ?ndice
\tableofcontents % Genera la tabla de contenidos del ?ndice autom?ticamente
\newpage

%Lista de figuras 
\listoffigures
\newpage

%Lista de tablas 
\listoftables
\newpage

\section{Introducción}

    Los \textbf{preámbulos de  la fe} consisten en algunas verdades preliminares que 
    sirven de fundamento al estudio de la religión. Estas verdades articulan la fe; 
    mas aquí las vamos a considerar únicamente a la luz de la razón y de la ciencia.

    Estas verdades podemos reducirlas a cinco principales:

    \begin{enumerate}
        \item Existe un Dios, creador de todos los seres.
        \item El hombre, creado por Dios, tiene un alma espiritual, libre e inmortal.
        \item Sólo puede haber una religión, buena y verdadera.
        \item La única religión verdadera es la cristiana.
        \item La verdadera religión cristiana, es la católica.
    \end{enumerate}

\end{document}
