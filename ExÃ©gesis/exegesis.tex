\documentclass[11pt, oneside, titlepage]{book}
\usepackage{fancyhdr}
\usepackage{graphicx}
\usepackage{imakeidx}
\usepackage{makeidx}
\usepackage{mathtools}
\usepackage[spanish]{babel}
\usepackage{graphicx}
\usepackage{dsfont}
 
\title{\textbf{Exégesis}}
\author{Francisco Javier Balón Aguilar}
%\date{}

\usepackage{amsmath}
\begin{document}
\maketitle
% Índice de contenidos
\tableofcontents
\newpage

\chapter{Apocalipsis}

    Revelación.
    Diálogo personal con Cristo.
    Apoteósis del Evangelio.
    Escrito por San Juan.

    \section{Contexto}
    Siglo I, después de la muerte y resurrección. Dificultades de las Iglesias de Asia (toda la Iglesia, 7 es plenitud, tanto Iglesia en comunidad como Iglesia ser personal de todos los tiempos).
    Combate y guerra espiritual -a veces física-. 

    Nicolaítas - herejes

    \section{Lenguaje}
    género literario propio, género apocalíptico
    todos los apocalípsis bíblicos tienen en común: lucha de Dios contra una fuerza mundana (imperio romano, anticristo...) --> Las Dos Ciudades
    habla al alma, al espíritu

    la tierra es contemplada desde el Cielo, muy interesante - Eucaristía (el día del Señor) arrebatado en espíritu

    libro profético, tiene que suceder pronto

    meditar y guardar (Madre Trinidad: vivir y recibir)

    \section{Estructura}

    \subsection{Las cartas a las 7 iglesias}

        trompeta --> tradición judía de aviso del Sabatt
        7 candelabros de oro --> menoráh (templo Santo de los Santos (protosagrario, presencia de Dios)), el templo destruido (año 70); la Iglesia como Cuerpo de Cristo sustituye la menoráh (7 brazos - 7 iglesias), templo físico - cuerpo místico
        descripción de Cristo Resucitado:
            hábitos reales (vestido largo) [capa de Rey?]
            hábitos sacerdotales (pecho de oro - pectoral de sacerdote)
            voz potente, espada de doble filo: autoridad (profeta)
            pies de bronce: indefectibilidad
            cabellos blancos: eternidad y venerabilidad
            ojos de llama: nada escapa a su mirada
            7 estrellas: poder sobre las 7 iglesias (sobre la plenitud de la Iglesia)
        soy el primero y el último, alfa omega, principio y fin

    \subsection{Carta a la Iglesia de Éfeso}

        Fundada por Pablo Juan. En Éfeso vivió María. Iglesia madre de las Iglesias de Asia (y toda la Iglesia).

        Esto dice el que \textbf{gobierna}. Las Iglesias son el descanso de Dios (o deberían, Cristo se pasea).

        6 alabanzas (fatiga, perseverancia, no soporta malvados, a prueba apostoles que no son, son mentirosos...).

        1 reproche: has abandonado tu amor primero --> formalismo, doctrina vacía

        remover tu candelabro: no eres Iglesia

    \subsection{Carta a la Iglesia de Esmirna}

        Iglesia que sufre calumnias y es perseguida.

        Esmirna es ciudad rica y próspera. Hostil a los cristianos (Obispo Policarpo asesinado en Esmirna por paganos y judíos)

        No ceder ante la intimidación de los que se llaman judíos.

    \subsection{Carta a la Iglesia de Pérgamo}
    \subsection{Carta a la Iglesia de Tiatira}
    \subsection{Carta a la Iglesia de Sardis}
    \subsection{Carta a la Iglesia de Filadelfia}
    \subsection{Carta a la Iglesia de Laodicea}

    \section{La Liturgia Celeste}
    \section{Las Bodas del Cordero}

% BIBLIOGRAFÍA
%Suma Teologica de Santo Tomás de Aquino
%Bibliografía de Cardenal Robert Sarah
%Estudios de Doña Beatriz Ozores (Apocalipsis)

\end{document}
