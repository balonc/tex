\documentclass[11pt, oneside, titlepage]{article}
\usepackage{fancyhdr}
\usepackage{graphicx}
\usepackage{imakeidx}
\usepackage{makeidx}
\usepackage{mathtools}
\usepackage[spanish]{babel}
\usepackage{graphicx}
\usepackage{dsfont}
 
\title{{\scshape\Huge Fundamentos Matemáticos \par}}
\author{Francisco Javier Balón Aguilar}
\date{2021}

\usepackage{amsmath}
\begin{document}
\maketitle
% Índice de contenidos
\tableofcontents
\newpage

\section{Prólogo}

    El presente documento no pretende ser guía, ni su orden se encuentra definido por un 
    calendario definido. Más bien, simplemente son los apuntes del autor del documento, usados para,
    mejor que preparar, fundamentar y consolidar los conocimientos matemáticos que considera 
    básicos para poder adentrarse en ingeniería sin muchos problemas.

    Por esta razón y al tratarse de unas hojas de apuntes fundamentalmente, el contenido aquí 
    descrito podrá ser excesivo para muchos, o básicos e inútiles para otros.

\newpage

\section{El Número Real}

\end{document}
