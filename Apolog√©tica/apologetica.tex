\documentclass[11pt, oneside, titlepage]{book}
\usepackage{fancyhdr}
\usepackage{graphicx}
\usepackage{imakeidx}
\usepackage{makeidx}
\usepackage{mathtools}
\usepackage[spanish]{babel}
\usepackage{graphicx}
\usepackage{dsfont}
 
\title{\textbf{Apologética Científica}}
\author{Francisco Javier Balón Aguilar}
%\date{}

\usepackage{amsmath}
\begin{document}
\maketitle
% Índice de contenidos
\tableofcontents
\newpage

\chapter{En torno a la metafísica de la materia}

    \section{El mundo de la materia <<no-viviente>>}
    \section{Objetividad de los sentidos}
    \section{Estructura espacial del mundo}
    \section{El movimiento}
    \section{El tiempo}
    \section{Actividad de la materia}
    \section{Constitución de la materia}
    \section{Origen del universo}
    \section{Futuro del universo}
    \section{Límites del conocimiento}
    \section{Apuntes varios}

        \begin{quotation}
            \small <<El primer sorbo del vaso de las ciencias naturales te convertirá en ateo, 
            pero en el fondo del vaso te aguarda Dios>>. W. Heisenberg, premio nobel en física 
            cuántica.
        \end{quotation}

        A la que añade: <<Los caminos misteriosos de Dios son extremadamente parecidos a la 
        mecánica cuántica.>>

\chapter{En torno al origen de la vida}

    El origen de la vida ha sido un motivo permanente de reflexión por parte del pensamiento especulativo 
    de todos los tiempos.
    
    Como es bien sabido, desde la más remota antigüedad y hasta hace relativamente poco, apenas un siglo, 
    se pensaba que la vida podía originarse en forma espontánea, a partir de la materia inanimada. Toda 
    experiencia parecía confirmarlo, siendo una conclusión perfectamente razonable y lógica de acuerdo a 
    los métodos de observación disponibles y a los conocimientos de la época. También es una conclusión 
    <<perfectamente equivocada>>, como el Dr. Raúl Osvaldo Leguizamón apunta, como hoy sabemos tras Louis 
    Pasteur, quien demostró --definitivamente-- que, bajo las condiciones actuales de la naturaleza, no 
    existe generación de vida en forma espontánea a partir de materia inanimada.

    ¿De dónde provino, pues, la primera manifestación de vida? Este es uno de los problemas más apasionantes 
    de la Biología en este momento histórico, incluso siendo una de las aspiraciones científicas detrás del 
    proyecto espacial.

    Es conveniente aclarar que, a pesar de los experimentos de Pasteur, actualmente muchos científicos
    siguen creyendo en la generación espontánea de la vida a partir de la materia inanimada, pero con una 
    nueva denominación más <<científica y elegante>>; \textit{arquebiopoyesis} o \textit{biogénesis primitiva}, 
    afirmándose que los experimentos de Pasteur sólo demostraron que la generación espontánea no ocurre ahora,
    pero no, que no haya ocurrido en el pasado bajo otras circunstancias. Hay que añadir que la nueva teoría
    difiere en dos aspectos: las condiciones ambientales y el tiempo de generación (siendo en la primera unas 
    pocas semanas y en la nueva unos millones de años)\footnote{
        La vieja hipótesis de era en realidad vitalista, ya que postulaba la existencia, en la intimidad de la materia, 
        de ciertas “fuerzas vegetativas”, que en determinado momento producían la vida. La
        espontaneidad se refería sólo a la manifestación de la vida, no a su origen, el cual se atribuía a esas fuerzas
        seminales y no a la materia inanimada en sentido estricto.    
    }. George Wald, premio Nobel de 
    Bioquímica y profesor de la Universidad de Harvard da forma a este argumento:

    \begin{quote}
        \textit{
            <<Pienso que un científico no tiene otra opción que abordar el origen de la vida a través de una
            hipótesis de generación espontánea... (lo que Pasteur) demostró ser insostenible, es sólo la creencia
            de que los organismos vivientes se originan espontáneamente en las condiciones actuales>>.    
        }
    \end{quote}

    Incurriendo después en lo que podríamos denominar como una contradicción:

    \begin{quote}
        \textit{
            <<Uno sólo tiene que contemplar la magnitud de esta tarea (evolución de la vida primitiva a
            partir de sustancias inorgánicas), para conceder que la generación espontánea de un organismo viviente
            es imposible. Y sin embargo aquí estamos, como resultado --creo yo-- de la generación espontánea>>. 
            \footnote{
                George Wald, “The Origin of Life”, Sci. Amer., 191, 45 (1954), p. 46.
            }   
        }
    \end{quote}

    Ante esta frase el Dr. Raúl Osvaldo Leguizamón afirma con fiereza ante esta declaración: \textit{<<
    Como se ve, para algunos científicos, la imposibilidad de un fenómeno no afecta a su credibilidad>>}.

    \section{¿Qué es, pues, la vida?}

        Sin entrar a definiciones filosóficas, podemos concluir que en Ciencia no hay una definición formalmente 
        aceptada de la <<vida>>, idea de Platón, ya que es muy difícil, si no imposible, de precisar en términos 
        científicos rigurosos.

        Una definición de <<ser vivo>>, aparentemente contraria con el consenso de la mayoría de los científicos, 
        tomanda de la magistral obra \textit{Azar y certeza}, del biólogo y matemático francés Georges Salet, es 
        la siguiente:

        \begin{quote}
            \textit{
                <<Un ser vivo es un ensamblado material autónomo, donde se realizan intercambios energéticos y
                químicos con el medio ambiente, ordenados a la asimilación, reproducción y adaptación>>.
                \footnote{
                    Georges Salet, Azar y certeza, ed. Alhambra, 1975, p. 38.
                }
            }
        \end{quote}

        No existe ningún ser vivo que no cumpla con estos criterios. No existe nada que cumpla con estos
        criterios y que no sea un ser vivo.\footnote{
            Como se ve, una definición estrictamente mecanicista, o mejor, maquinicista. En el sentido de que
            no es vitalista.
        }   

% En torno al origen de la vida - Raúl Osvaldo Leguizamón

\end{document}
