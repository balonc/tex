\documentclass[11pt, oneside, titlepage]{article}
\usepackage{fancyhdr}
\usepackage{graphicx}
\usepackage{imakeidx}
\usepackage{makeidx}
\usepackage{mathtools}
\usepackage[spanish]{babel}
\usepackage{graphicx}
\usepackage{dsfont}
 
\title{\textbf{Pokémon Classic}}
\author{Francisco Javier Balón Aguilar}
%\date{}

\usepackage{amsmath}
\begin{document}
\maketitle
% Índice de contenidos
% \tableofcontents
% \newpage

Cuando \textit{Satoshi Tajiri} era joven, uno de sus pasatiempos era la recolección y 
colección de insectos. \textit{Tajiri} se dirigió a \textit{Tokio} a estudiar, ya que su 
padre quería que fuese ingeniero. Sin embargo, al joven \textit{Tajiri} no le agradaba la 
idea de estudiar y pasaba más tiempo jugando a videojuegos. Posteriormente, llegó a trabajar 
como \textit{probador de videojuegos} para algunas revistas junto a \textit{Ken Sugimori}, 
con quien hizo una gran amistad. En 1989 crearon la revista \textit{Game Freak}.

Con el éxito de la consola \textit{NES} ambos decidieron crear algo innovador para ésta, 
convirtiendo \textit{Game Freak} en una compañía de videojuegos por decisión de \textit{Tajiri}.
Al año siguiente de su primer juego\footnote{
    Un rompecabezas llamado \textit{Mendel Palace}, conocido en Japón como \textit{Quinty}. 
    Lanzado en 1989 obteniendo buen éxtivo, lo que marcaría el principio de la historia de 
    \textit{Game Freak} como compañía.
}, \textit{Tajiri} y \textit{Sugimori} decidieron crear un juego para la consola \textit{Game Boy}.
\textit{Tajiri}, al ver el \textit{Game Link Cable} ideó un juego donde se pudiera transferir 
información de una \textit{Game Boy} a otra e, influenciado por sagas como \textit{Final Fantasy}
y \textit{Dragon Quest} asoció la idea de metamorfosis. Se creó pues un RPG donde los 
monstruos podrían evolucionar y ser trasportados de una consola a otra.

El proyecto fue enviado a \textit{Nintendo}. Mientras que \textit{Tajiri} era quien tenía 
la idea principal, \textit{Sugimori} era el encargado de los diseños de los monstruos. Así mismo, 
recibieron consejos por parte de \textit{Shigeru Miyamoto}\footnote{
    Creador de Mario Bros.
} para mejorar el juego, que en ese entonces recibía el nombre de \textit{Capsule Monster}.

\end{document}
