\documentclass[11pt, twosides, titlepage]{article}
\usepackage{fancyhdr}
\usepackage{graphicx}
\usepackage{imakeidx}
\usepackage{makeidx}
\usepackage{mathtools}
\usepackage[spanish]{babel}
\usepackage{graphicx}
\usepackage{amssymb}
\usepackage{listings}
\usepackage{xcolor}
\usepackage{hyperref}

\setcounter{secnumdepth}{5}
\setcounter{tocdepth}{5}

\title{{\scshape\Huge Puertos Serie en GNU/Linux\par}}
\author{Francisco Javier Balón Aguilar}

\begin{document}
\maketitle
\renewcommand{\contentsname}{Índice de contenidos} % Nombre dado al ?ndice
\tableofcontents % Genera la tabla de contenidos del ?ndice autom?ticamente
\newpage

%Lista de figuras 
\listoffigures
\newpage

%Lista de tablas 
\listoftables
\newpage

\section{Introducción}

    El puerto serie convencional (no el puerto USB más nuevo ni el puerto Firewire) es un 
    puerto de E/S muy antiguo. Hasta alrededor de 2006, la mayoría de las PC de escritorio 
    nuevas tenían una, y las PC antiguas de la década de 1990 a veces tenían 2 de ellas. 
    La mayoría de las computadoras portátiles las abandonaron antes que las computadoras 
    de escritorio. Los Apple Macs después de mediados de 1998 solo tenía el puerto 
    USB. Sin embargo, es posible poner un dispositivo de puerto serie convencional en el 
    bus USB que está en todos los PC modernos.

    Cada puerto serie tiene un <<archivo>> asociado en el directorio \textbf{/dev}. No es 
    realmente un archivo, pero aparenta ser uno. Por ejemplo, \textbf{/dev/ttyS0}. Otros 
    puertos serie \textbf{/dev/ttyS1}, \textbf{/dev/ttyS2}, etc. 

    La denominación común para el puerto serie convencional es RS-232, a menudo 
    llamado "puerto serie RS-232". Los conectores para el puerto serie generalmente se ven 
    como uno o dos conectores de 9 pines (en algunos casos de 25 pines). Pero el puerto 
    serie es más que solo conectores. Incluye la electrónica asociada que debe producir 
    señales conforme a la especificación RS-232. Véase \ref{ondas_voltage}. Un pin se 
    utiliza para enviar bytes de datos y otro para recibir bytes de datos. Otro pin es 
    una señal de tierra común. Los otros pines <<útiles>> se utilizan principalmente para 
    fines de señalización con un voltaje negativo constante que significa <<apagado>> y un 
    voltaje positivo constante que significa <<encendido>>.

    El chip UART (\textit{Universal Asynchronous Receiver-Transmitter}) hace la mayor parte 
    del trabajo. Hoy en día, la funcionalidad de este chip generalmente se integra en otro 
    chip. Véase \ref{uarts}. Estos han mejorado con el tiempo y los modelos antiguos (antes 
    de 1994) suelen ser muy obsoletos.

\section{Cómo los bytes son transmitidos por hardware}

    \subsection{Transmisión}

        La transmisión consiste en enviar bytes fuera del puerto serie lejos de la 
        computadora (salida). Recibir (entrada) es fácil de asimilar. La primera 
        explicación dada aquí será simplificada. Luego se agregarán más detalles en 
        explicaciones posteriores (véase \ref{fifos}). Cuando la computadora desea enviar 
        un byte al puerto serie (al cable externo), la CPU envía el byte en el bus dentro 
        de la computadora a la dirección de E/S\footnote{
            E/S a menudo se escribirá como IO.
        } del puerto serie. El puerto serie toma el byte y lo envía un bit a la vez 
        (un flujo de bits en serie) en el pin de transmisión del conector de cable en 
        serie. Para lo que un poco (y byte) parecen eléctricamente ver Formas de onda de 
        voltaje.

        Aquí hay una repetición de lo anterior con un poco más de detalle (pero aún muy 
        incompleta). La mayor parte del trabajo en el puerto serie lo realiza el chip 
        UART (o similar). Para transmitir un byte, el programa de controlador de 
        dispositivo serie (que se ejecuta en la CPU) envía un byte a la dirección de E/s 
        del puerto serie. Este byte entra en un "registro de cambio de transmisión" de 1 
        byte en el puerto serie. A partir de este registro de desplazamiento, los bits se 
        toman del byte uno por uno y se envían bit a bit en la línea serie. Luego, cuando 
        se ha enviado el último bit y el registro shift necesita otro byte para enviar, 
        simplemente podría pedirle a la CPU que le envíe otro byte. Por lo tanto sería 
        simple, pero probablemente introduciría retrasos ya que la CPU podría no ser capaz 
        de obtener el byte inmediatamente. Después de todo, la CPU generalmente está 
        haciendo otras cosas además de manejar el puerto serie.
        
        Una forma de eliminar tales retrasos es organizar las cosas para que la CPU 
        obtenga el byte antes de que el registro de cambios lo necesite y lo almacene en 
        un búfer de puerto serie (en hardware). Luego, cuando el registro shift ha enviado 
        su byte y necesita un nuevo byte inmediatamente, el hardware del puerto serie solo 
        transfiere el siguiente byte de su propio búfer al registro shift. No es necesario 
        llamar a la CPU para obtener un nuevo byte.

\end{document}
