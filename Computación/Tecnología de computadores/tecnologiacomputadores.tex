\documentclass[a4paper, 11pt, titlepage]{article}
\usepackage{fancyhdr}
\usepackage{graphicx}
\usepackage{imakeidx}
\usepackage{makeidx}
\usepackage{mathtools}
\usepackage[spanish]{babel}
\usepackage{eurosym}
\usepackage{hyperref}
\usepackage{amssymb}
\usepackage{listings}
\usepackage{xcolor}

\title{Tecnología de computadores}
\author{Francisco Javier Balón Aguilar}

\begin{document}

\maketitle
\renewcommand{\contentsname}{Índice}
\tableofcontents
\newpage

\section{Introducción}

    \subsection{Transformación de valores analógicos a valores digitales}

        El mundo físico se caracteriza por continuas, el cambio de valor se realiza de forma continua, hay infinitos 
        valores con un posible infinito número de decimales. Por lo que para poder medir y trabajar con magnitudes 
        analógicas se discretizan, es decir, los posibles valores se reducen a unos pocos, transformándolo en magnitudes 
        digitales (interpretables mediante ordenadores). Estas medidas discretizadas se realizan, además, cada cierto 
        tiempo y no de forma contínua, muestreando los valores de entrada.

        Para realizar estas transformaciones se utilizan conversores de magnitudes analógicas a digitales o A/D y de 
        magnitudes digitales a analógicas o D/A, contando con las siguientes etapas de conversión:

        \begin{itemize}
            \item \textbf{Muestreador.} Muestrea la señal analógica de entrada para discretizar el número de medidas 
            tomadas.
            \item \textbf{Cuantificador.} Asigna a la magnitud analógica una magnitud digital. Inevitablemente se 
            añadirá ruido a la señal.
            \item \textbf{Codificador.} Transforma las magnitudes digitales, generalmente, a código binario.
        \end{itemize}

    \subsection{Sistemas de numeración}

        Los sistemas de numeración se utilizan para representar magnitudes. Existen multitud de ellos, aunque el más 
        utilizado de forma cotidiana por los humanos es el sistema decimal, llamado así por tener base 10, es decir, 
        emplear 10 dígitos:

        \[[0,1,2,3,4,5,6,7,8,9]\]

        En el caso del computador el sistema más utilizado es el sistema binario, de base 2:

        \[[0,1]\]

        Otro sistema de numeración que suele emplearse en el estudio de la computación es el 
        sistema hexadecimal, de base 16:

        \[[0,1,2,3,4,5,6,7,8,9,A,B,C,D,E,F]\]

        Este sistema sólo se utiliza en el estudio, como se ha dicho, porque permite expresar de 
        forma compacta secuencias de 0s y 1s, acercando el binario al humano.

        Cabe destacar que los sistemas de numeración pueden ser combinados para construir otros 
        sistemas más complejos. Un ejemplo de ello es el sistema de representación en coma 
        flotante de números decimales.

    \subsection{Representación de números enteros}

        \subsubsection{Binario}

        \subsubsection{Signo-binario}

        \subsubsection{Complemento a 1}

        \subsubsection{Complemento a 2}

        \subsubsection{Exceso a Z}

    \subsection{Representación de números reales}

    \subsection{Representación de caracteres alfanuméricos}

        \subsubsection{ACII extendido}

        \subsubsection{EBCDIC}

\end{document}