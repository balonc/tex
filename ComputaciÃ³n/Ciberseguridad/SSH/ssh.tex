\documentclass[a4paper, 11pt, titlepage]{article}
\usepackage{fancyhdr}
\usepackage{graphicx}
\usepackage{imakeidx}
\usepackage{makeidx}
\usepackage{mathtools}
\usepackage[spanish]{babel}
\usepackage{eurosym}
\usepackage{hyperref}
\usepackage{amssymb}
\usepackage{listings}
\usepackage{xcolor}

\setcounter{secnumdepth}{5}
\setcounter{tocdepth}{5}

\title{Robustecimiento y securización SSH}
\author{Francisco Javier Balón Aguilar}

\begin{document}

\maketitle
\renewcommand{\contentsname}{Índice}
\tableofcontents
\newpage

\section*{Prólogo}
\newpage

\section{Introducción a SSH}

    SSH --o Secure Shell-- es uno de los protocolos más interesantes de los que se dispone 
    en la administración de sistemas, permitiendo acceder a otros equipos de forma remota; 
    pero sobretodo, de forma segura.
    
    Con SSH creamos un túnel entre cliente y servidor que simula, en la capa de aplicación, 
    la shell o intérprete de comandos del servidor, dotando al administrador de prácticamente 
    toda funcionalidad en el equipo.

    Generalmente se suele equiparar el protocolo SSH con el protocolo Telnet, ya que ambos 
    pretenden una funcionalidad similar. Aun así, existen diferencias notables, esencialmente 
    en la seguridad de las comunicaciones; ya que SSH utiliza técnicas criptográficas para 
    proteger los datos intercambiados entre ambos nodos que conforman la conexión SSH. SSH 
    utiliza un intercambio de claves basado en el protocolo criptográfico de \textit{Diffie-Hellman}, 
    imprescindible en el robustecimiento de la capa de transporte.

    Como vemos, SSH proporciona un sistema, a priori, seguro. Con una configuración con un 
    nivel aceptable de seguridad por defecto. Pero en función de la configuración del 
    servicio puede llevar al sistema a un estado de inseguridad. Por ello, es importante 
    entender el funcionamiento del protocolo, las funcionalidades que presenta y la configuración
    óptima en función de las necesidades del administrador. Este será el núcleo del presente 
    documento.

\section{Visión técnica del protocolo}
\section{Configuración del servicio}
    \subsection{Directivas básicas}
    \subsection{Autenticación básica con clave}
    \subsection{Autenticación mediante par de claves}
    \subsection{Proceso de conexión}
\section{Aplicaciones de SSH}
    \subsection{SCP}
    \subsection{SFTP}
    \subsection{SSHFS}
    \subsection{X11 forwarding}
    \subsubsection{Fail2ban}
\section{Tunneling}
    \subsection{Túneles TCP/IP con port forwarding}
\section{SOCKS}

\end{document}