\documentclass[a4paper, 11pt, titlepage]{article}
\usepackage{fancyhdr}
\usepackage{graphicx}
\usepackage{imakeidx}
\usepackage{makeidx}
\usepackage{mathtools}
\usepackage[spanish]{babel}
\usepackage{eurosym}
\usepackage{hyperref}
\usepackage{amssymb}
\usepackage{listings}
\usepackage{xcolor}

\setcounter{secnumdepth}{5}
\setcounter{tocdepth}{5}

\title{\textbf{emacs}\\ Documentación técnica}
\author{Francisco Javier Balón Aguilar}

\begin{document}

\maketitle
\renewcommand{\contentsname}{Índice}
\tableofcontents
\newpage

\section{emacs}

    GNU emacs posiblemente sea el editor de textos más potente que exista 
    para sistemas UNIX, lo cual es tanto como decir que se trata del 
    editor de textosmás potente que existe en términos absolutos.

    Tiene una serie de características que lo hacen único:

    \subsubsection{Reconocimiento de formatos} Es una herramienta esencial en 
    el editor. Esta característica permite a emacs detectar que cierto fichero 
    sigue determinadas convenciones de sintaxis, generalmente correspondiente a 
    un lenguaje de programación o un lenguaje de marcas, de tal modo que, una vez 
    reconocido emacs pueda proporcionar una serie de comandos y mandatos útiles 
    para este tipo de documento, así como resaltar mediante procedimientos gráficos 
    la sintaxis, distinguiendo las instrucciones y los datos; e incluso 
    diferenciando entre distintas categorías de instrucciones.

    Si bien es cierto que muchos editores actuales han implementado esta habilidad, 
    emacs, que fue de los primeros en implementarla, sigue siendo capaz de reconocer 
    más formatos que la mayoría. Además su diseño basado en <<modos mayores de edición>>
    se ajusta de manera más natural al manejo de diferentes tipos de ficheros.

    \subsubsection{Flexibilidad de configuración y personalización} Podemos decir, 
    que en emacs prácticamente todo es personalizable. Podemos crear nuevos comandos, 
    asignar nuevas combinaciones de teclas, alterar variables propias que nutren el 
    funcionamiento del editor, etc. 

    \subsubsection{Extensibilidad} La facilidad de configurar emacs queda ampliamente 
    superada por su extensibilidad. Decir que emacs es extensible significa que 
    cualquiera que sepa emacs Lisp\footnote{emacs Lisp es un dialecto de Lisp en el que está escrito 
    la mayor parte del propio emacs.} puede escribir nuevos comandos de emacs, incorporarlos al 
    sistema, sobre la marcha, sin necesidad de reinstalar o reiniciar el propio editor.

    Esta facilidad para extender y ampliar el sistema se traduce en que existen numerosos 
    paquetes de ampliación que permiten reconocer nuevos formatos, o le dotan de comandos 
    concretos para ciertos formatos o ciertos usos; haciendo que, en definitiva, emacs 
    sea útil para todo. Existen paquetes que convierten emacs en un lector de correo, 
    un lector de noticias, un entorno integrado de desarrollo, un calendario, etc.

    \subsubsection{Dura curva de aprendizaje} Sin embargo, emacs no es un programa sencillo.
    Su extremada potencia hace que sean muchas las funciones y comandos que hay que aprender.
    Además emacs es muy peculiar, siendo un software de la vieja escuela, lo que hace que 
    su terminología y conceptos no se asemejen a los de otras aplicaciones y estándares de 
    finalidad similar.

    \subsubsection{Los comandos emacs} emacs, como mencionamos anteriormente, 
    funciona mediante la ejecución de comandos: abrir o cerrar un fichero, insertar 
    texto o borrarlo, desplazar el cursor, buscar en el texto... Todo se realiza 
    mediante comandos. Incluso escribir una letra es fruto de un comando.

    Desde el punto de vista interno los comandos de emacs son funciones, generalmente 
    escritas en Elisp; pero desde el punto de vista del usuario son comandos propiamente 
    denominados, es decir, acciones que se realizan al pulsar determinada combinación de 
    teclas o seleccionamos una opción de menú.

    \paragraph{Formas de invocar comandos} Existen básicamente tres formas de ejecutar 
    un comando:
    
    \begin{enumerate}
        \item Invocar el comando por su nombre.
        \item Pulsar en el teclado la combinación de teclas a la que cierto mandato esté 
        asociada.
        \item Seleccionar el mandato del menú o de la barra de herramientas.
    \end{enumerate}

\end{document}