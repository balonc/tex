\documentclass[a4paper, 11pt, titlepage]{article}
\usepackage{fancyhdr}
\usepackage{graphicx}
\usepackage{imakeidx}
\usepackage{makeidx}
\usepackage{mathtools}
\usepackage[spanish]{babel}
\usepackage{eurosym}
\usepackage{hyperref}
\usepackage{amssymb}
\usepackage{listings}
\usepackage{xcolor}

\setcounter{secnumdepth}{5}
\setcounter{tocdepth}{5}

\title{\textbf{Software de gestión de información confidencial basado en Python}\\ 
Trabajo Fin de Grado Superior en Desarrollo de Aplicaciones Multiplataforma}
\author{Francisco Javier Balón Aguilar}
\date{21 de junio de 2018}

\begin{document}

\maketitle
\renewcommand{\contentsname}{Índice}
\tableofcontents
\newpage

\section{Estudio preliminar del problema}

    El presente documento detalla la el desarrollo del proyecto denominado Securebox, 
    realizado como proyecto integrado para la obtención del título de Técnico en 
    Desarrollo de Aplicaciones Multiplataforma por parte del centro Grupo Studium, 
    Sevilla. 

    El objetivo del proyecto consiste en desarrollar un software multiplataforma que 
    permita la gestión y almacenamiento de información, garantizando principalmente 
    la confidencialidad y la integridad de la información utilizando criptografía. 

    El software desarrollado soportará la gestión de usuarios, pudiendo existir 
    varios usuarios en la misma base de datos, siendo cada uno de ellos propietario 
    de diferentes secretos. 

    \subsection{Funciones básicas}

        \subsubsection{Control de acceso}

            El software almacena usuarios y contraseñas para realizar su función. 
            Es una función indispensable en el sistema, ya que cada fichero e 
            información irá vinculada directamente al usuario. Securebox es multiusuario, 
            en un equipo podrá haber varios usuarios pero sólo uno mantendrá una sesión 
            iniciada. A su vez, el sistema permite la creación de usuarios.

        \subsubsection{Gestión de información secreta}

            El núcleo lógico de Securebox es la gestión de información secreta. Esta 
            se almacenará en la base de datos junto a diferentes metadatos que permitan 
            el correcto desempeño de la aplicación. Podemos clasificar la información 
            secreta en archivos binarios y cadenas de texto.

            Los archivos binarios son ficheros de cualquier tipo y extensión. Pueden 
            ser imágenes, documentos, ficheros de texto, audio, vídeo, etc. Estos se 
            almacenarán en el sistema.

            Al igual que los archivos, las cadenas de texto permiten la gestión de 
            información introducida directamente por parte del usuario iniciado, de 
            forma que pueda almacenar de forma segura mensajes secretos, contraseñas, 
            pins, etc. Que serán almacenados en la base de datos.

    \subsection{Viabilidad del sistema}

        El lenguaje de programación orientado a objetos Python, utilizado como base 
        del sistema, crece en cuota de mercado a gran velocidad. Estimándose como uno 
        de los lenguajes más utilizados en el futuro próximo de la tecnología, lo que 
        asegura un continuo soporte y comunidad. Estos factores favorecen un amplio 
        periodo en la línea temporal en el que esta aplicación podrá ser utilizada sin 
        quedar obsoleta, algo a tener en cuenta en un mundo tecnológico tan mutable.

\end{document}