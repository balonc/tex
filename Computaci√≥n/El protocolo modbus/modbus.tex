\documentclass[11pt, twosides, titlepage]{article}
\usepackage{fancyhdr}
\usepackage{graphicx}
\usepackage{imakeidx}
\usepackage{makeidx}
\usepackage{mathtools}
\usepackage[spanish]{babel}
\usepackage{graphicx}
\usepackage{amssymb}
\usepackage{listings}
\usepackage{xcolor}
\usepackage{hyperref}

\setcounter{secnumdepth}{5}
\setcounter{tocdepth}{5}

\title{{\scshape\Huge El protocolo modbus RTU\par}}
\author{Francisco Javier Balón Aguilar}

\begin{document}
\maketitle
\renewcommand{\contentsname}{Índice de contenidos} % Nombre dado al ?ndice
\tableofcontents % Genera la tabla de contenidos del ?ndice autom?ticamente
\newpage

%Lista de figuras 
\listoffigures
\newpage

%Lista de tablas 
\listoftables
\newpage

\section{Introducción}

    El protocolo modbus es un protocolo de transmisión en serie, los datos se transmiten uno detrás de 
    otros. Comúnmente se implementa sobre redes de comunicaciones RS-485, pero también puede ser implementado
    sobre redes de comunicaciones RS-232 e incluso redes ethernet TCP/IP.

\section{Modos de transmisión}

    modbus cuenta con dos modos de transmisión:

    \begin{itemize}
        \item \textbf{RTU} o Remote Terminal Unit, donde la comunicación entre dispositivos se realiza 
        por medio de datos binarios.
        \item \textbf{ASCII} o American Standard Code for Information Interchange, donde la comunicación 
        entre dispositivos se realiza por medio de caracteres ASCII.
    \end{itemize}

    En el presente documento nos centraremos en el modo de transmisión RTU.

\end{document}
