\documentclass[a4paper, 11pt, titlepage]{article}
\usepackage{fancyhdr}
\usepackage{graphicx}
\usepackage{imakeidx}
\usepackage{makeidx}
\usepackage{mathtools}
\usepackage[spanish]{babel}
\usepackage{eurosym}
\usepackage{hyperref}
\usepackage{amssymb}
\usepackage{listings}
\usepackage{xcolor}

% \setcounter{secnumdepth}{5}
% \setcounter{tocdepth}{5}

\title{{\scshape\Huge Sistemas Operativos Distribuidos \par}}
\author{Francisco Javier Balón Aguilar}

\begin{document}
\maketitle
\renewcommand{\contentsname}{Índice de contenidos} % Nombre dado al ?ndice
\tableofcontents % Genera la tabla de contenidos del ?ndice autom?ticamente
%\newpage

%Lista de figuras 
\listoffigures
%\newpage

%Lista de tablas 
\listoftables
\newpage

\section{Introducción}

    \subsection{Estructura de un sistema distribuido}

        \subsubsection{Funciones de un sistema distribuido}

        \subsubsection{Características de un sistema distribuido}

    \subsection{Tipos de sistemas operativos distribuidos}

\section{Sistemas operativos multiprocesador}

    \subsection{Arquitectura de un sistema operativo multiprocesador}

        Los equipos que cuentan con más de un procesador y que comparten la misma memoria
        se denominan sistemas multiprocesador. En ellos, debe existir un sistema operativo 
        común y central, que controle las operaciones de cada procesador, así como su 
        gestión, coordinación y accesos a memoria.

        Este tipo de sistemas proporcionan una mayor productividad del sistema, ya que 
        pueden realizar un mayor número de tareas en un menor espacio de tiempo, y una 
        mayor velocidad de aplicación, ya que al aumentarse el número de procesadores,
        disminuye el tiempo de espera de un proceso para ser procesado.

        Estos sistemas ofrecen una serie de ventajas:

        \begin{itemize}
            \item \textbf{Rendimiento y potencia de cálculo}.
            \item \textbf{Tolerancia a fallos}.
            \item \textbf{Flexibilidad}.
            \item \textbf{Relación coste/rendimiento}.
        \end{itemize}

        \subsubsection{Interconexión de los procesadores}

    \subsection{Gestión del procesador}

        \subsubsection{Asignación de los procesadores}

        \subsubsection{Planificación del procesador}

    \subsection{Sincronización y gestión de la memoria}

        \subsubsection{Algoritmos de sincronización}

% BIBLIOGRAFÍA Y REFERENCIAS
\newpage
\begin{thebibliography}{X}
    \bibitem{} Desarrollo de un sistema expertocon lógica difusa, Jorge Franco Herrera y Angélica Franco Arias \\ \url{https://ingsistycomp.files.wordpress.com/2017/09/proyecto-1-sistema-experto-difuso.pdf}
    \bibitem{} Diseño de un Sistema Experto Difuso para la Determinación de la Densidad de Corriente en una Planta de Cromado, Carolina V. Ponce y Bayron Rojas \\ \url{https://scielo.conicyt.cl/scielo.php?script=sci_arttext&pid=S0718-07642019000200157}
    \bibitem{} Modelo basado en Lógica Difusa para el Diagnóstico Cognitivo del Estudiante \\ \url{https://scielo.conicyt.cl/scielo.php?script=sci_arttext&pid=S0718-50062012000100003}
    \bibitem{} Sistemas Expertos y Lógica Difusa \\ \url{http://catarina.udlap.mx/u_dl_a/tales/documentos/lmt/maza_c_ac/capitulo2.pdf}
    \bibitem{} Sistema de Control Difuso para Unidades de Cuidados Intensivos (UCI), Jefferson Steven Soto Medellín \\ \url{https://repository.ucatolica.edu.co/bitstream/10983/1278/1/Sistemas%20de%20Control%20Difuso%20para%20Unidades%20de%20Cuidado%20Intensivo%20(Trabajo%20Final)%20701429%20Nuevo.pdf}
\end{thebibliography}

\end{document}
