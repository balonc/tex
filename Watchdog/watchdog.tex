\documentclass[a4paper, 11pt, titlepage]{article}
\usepackage{fancyhdr}
\usepackage{graphicx}
\usepackage{imakeidx}
\usepackage{makeidx}
\usepackage{mathtools}
\usepackage[spanish]{babel}
\usepackage{eurosym}
\usepackage{hyperref}
\usepackage{amssymb}
\usepackage{listings}
\usepackage{xcolor}

\setcounter{secnumdepth}{5}
\setcounter{tocdepth}{5}

\title{\textbf{Watchdog}\\ Documentación técnica}
\author{Francisco Javier Balón Aguilar}

\begin{document}

\maketitle
\renewcommand{\contentsname}{Índice}
\tableofcontents
\newpage

\section{Watchdog}

    El watchdog (perro guardían o vigilante) del kernel de Linux se utiliza para 
    monitorizar y vigilar un sistema que se está ejecutando, de forma que pueda 
    ser reiniciado automáticamente en caso de fallo de software irrecuperables.

    El módulo de watchdog es específico para el hardware o chip que se utiliza.

    Los usuarios, por norma general, no necesitan de vigilancia en sus sistemas, 
    pues pueden ellos mismos restablecer el sistema manualmente; así está pensado
    para dispositivos críticos que requieran de restablecimientos sin intervención
    humana, por ejemplo servidores en una ubicación remota o equipos integrados 
    en naves espaciales.

    En su intervención es necesario proceder con precaución, ya que configuraciones
    incorrectas o erróneas podrían provocar:

    \begin{itemize}
        \item Bucles infinitos de reinicios.
        \item Corrupción de archivos debido al hard reset.
        \item Reinicios aleatorios impredecibles.
    \end{itemize}

    Por lo que es preferible evitar el uso de servidores en vivo para probar el 
    funcionamiento de watchdog.

    \subsection{Watchdog module}

        La funcionalidad de watchdog sobre hardware configura un temporizador 
        que agota el tiempo de espera después de un período predeterminado.
        El software de watchdog actualiza periódicamente el temporizador del 
        hardware, y si éste deja de actualizarse luego del período establecido 
        el temporizador realizará un restablecimiento de hardware del dispositivo.
        Para que un temporizador watchdog sea funcional, el fabricante de la placa 
        base tiene que permitir esta funcionalidad en el chip.

        A menudo, la documentación del fabricante de hardware no deja claro 
        sobre si se implementó o no esta funcionalidad. En ese caso hay que 
        testearlo directamente.

        Además es necesario el módulo del kernel de vigilancia correcto, que será 
        cargado en el sistema Linux. Diferentes chips utilizan diferentes módulos,
        por ejemplo:

        \begin{itemize}
            \item Intel utiliza \textit{iTCO\_wdt}.
            \item El hardware de HP utiliza \textit{hpwdt}.
            \item Los mainframes de IBM utilizan \textit{vmwatchdog}.
            \item Xen VM utiuliza \textit{xen\_wdt}.
        \end{itemize}

        Una vez corroborado la carga del módulo es posible verificar \textit{/dev/watchdog}
        en el sistema Linux. Si el archivo está presente, significa que se cargó el controlador.
        El sistema escribe periódicamente a \textit{/dev/watchdog} (esta acción
        suele llamarse coloquialmente <<patear o alimentar al perro guardián>> / <<kicking
        or feeding the watchdog>>). Si el sistema falla y no patea o alimenta al 
        perro, el sistema se restablecerá completamente.

    \subsection{Watchdog daemon}

        El demonio de watchdog abre el dispositivo y proporciona la actualización 
        necesaria para evitar que el sistema se restablezca. Éste puede comprobar 
        el espacio de la tabla de procesos, el uso de memoria (memory usage), la 
        accesibilidad de archivos, la carga de trabajo, desbordamientos en la tabla de 
        archivos, ping a direcciones IP, tráfico de interfaz de red, temperatura, 
        procesos en ejecución, etc. Como ya hemos visto, si las pruebas fallan 
        el perro provocará el apagado.

    \subsection{Iniciando y parando el demonio de watchdog}

        El demonio de watchdog debería iniciarse en el momento del arranque. 
        Podemos verificar si está iniciado con:

        \begin{lstlisting}[language=bash]
    ps -af | grep watch*\end{lstlisting}

        Si el kernel no estuviera compilado con \textit{CONFIG\_WATCHDOG\_NOWAYOUT}
        y cerramos correctamente \textit{/dev/watchdog} no habrá reinicio. Podemos 
        escribir un carácter $V$ en \textit{/dev/watchdog} y luego cerrar el archivo;
        lo que detendrá al perro guardián.

        \subsubsection{Testeando watchdog} Si queremos comprobar que el control de 
        hardware está funcionando, podemos lanzar la siguiente orden con privilegios 
        root:

        \begin{lstlisting}[language=bash]
    cat >> /dev/watchdog\end{lstlisting}

        Una vez haya pasado el tiempo esperado, dependiendo de la configuración 
        del kernel, el sistema realizará el hard reboot.

    \subsection{Configuración del demonio de watchdog}

        Una vez conocemos watchdog, podemos iniciar su configuración. El fichero de 
        configuración de watchdog se encuentra en \textit{/etc/watchdog.conf} 
        (es posible que haya que instalar el paquete \textit{watchdog} en la distribución 
        utilizada).

        \begin{lstlisting}[language=bash]
    # cat /etc/watchdog.conf
    realtime = yes
    priority = 1\end{lstlisting}

\end{document}